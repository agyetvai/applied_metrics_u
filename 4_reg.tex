\documentclass[aspectratio=169,compress,handout,t,xcolor=table]{beamer}

%! Mandatory packages
\usepackage[utf8]{inputenc}                                %! Character encoding
\usepackage[T1]{fontenc}                                   %! Font encoding
\usepackage[english]{babel}                                %! Language setting
\usepackage{graphicx}

%%%% Frame design
\usetheme{Boadilla}                                        % Design base
\usecolortheme{default}                                    % Color structure base

% Colors
\definecolor{MyBackground}{HTML}{FFFAEE}                   % Cream color with dark text works well for dyslexic readers: FFFAEE
\definecolor{MyBackground}{HTML}{FFFFFF}                   % White for online slides
\setbeamercolor{background canvas}{bg=MyBackground}
\definecolor{MyStructure}{HTML}{005471}
\setbeamercolor{title}{fg=MyStructure}
\setbeamercolor{frametitle}{fg=MyStructure}
\setbeamercolor{structure}{fg=MyStructure}

% Fonts
\usepackage{tgheros}                                       % TG Heros font package
\beamertemplatenavigationsymbolsempty                      % Disable navigation bar

% Enumerate / itemize design
\setbeamertemplate{enumerate items}[circle]
\setbeamertemplate{itemize items}[circle]
\setbeamertemplate{itemize subitem}[triangle]
\setbeamertemplate{section in toc}[sections numbered]
\usepackage{xpatch}                                        % Global itemize/enumerate list separation
\xpatchcmd{\itemize}
  {\def\makelabel}
  {\ifnum\@itemdepth=1\relax
     \setlength\itemsep{2ex}% separation for first level
   \else
     \ifnum\@itemdepth=2\relax
       \setlength\itemsep{1ex}% separation for second level
     \else
       \ifnum\@itemdepth=3\relax
         \setlength\itemsep{0.5ex}% separation for third level
   \fi\fi\fi\def\makelabel
  }
{}
{}


% Footer
\setbeamertemplate{footline}
{
 \hbox{%
  \begin{beamercolorbox}[wd=.2\paperwidth,ht=2.6ex,dp=1ex,center]{section in foot}%
    \usebeamerfont{section in foot}\insertshortauthor
  \end{beamercolorbox}%
  \begin{beamercolorbox}[wd=.75\paperwidth,ht=2.6ex,dp=1ex,center]{section in foot}%
    \usebeamerfont{section in foot}\insertsubtitle
  \end{beamercolorbox}%
  \begin{beamercolorbox}[wd=.05\paperwidth,ht=2.6ex,dp=1ex,center]{subsection in foot}%
    \usebeamerfont{section in foot} \insertframenumber/\inserttotalframenumber
  \end{beamercolorbox}}%

  \vskip0pt%
}
\setbeamertemplate{frametitle}[default][left]              % Push frame title to left
\setbeamertemplate{navigation symbols}{}                   % Remove navigation symbols

\renewcommand\textbullet{\ensuremath{\bullet}}             % Opress textbullet font warning
\usepackage{appendixnumberbeamer}                          % Restart slide numbering for backup slides (\appendix)

% Roadmap
\AtBeginSection[]
{
	\ifnum \value{framenumber}>0
   \begin{frame}[plain]
       \frametitle{Plan for today}
       \tableofcontents[currentsection]
   \end{frame}
   \addtocounter{framenumber}{-1}
  \else
   \fi
}

% Stable preamble with path
% Text packages
\usepackage{enumerate}                                     % Custom enumeration lists
\usepackage{footmisc}                                      % Stable footnotes in headings
\usepackage{natbib}                                        % Bibliography
\usepackage{hyperref}                                      % Hyperreferences
    \hypersetup{
      % colorlinks,
      % citecolor=black,
      % filecolor=black,
      % linkcolor=black,
      urlcolor=MyStructure
    }
\usepackage{multicol}                                      % Multiple columns in text
\usepackage[nointegrals]{wasysym}                          % WASY2 symbols for contradiction \lightning
\usepackage{algorithm}                                     % Pseudocode packages
\usepackage{algpseudocode}
\usepackage{setspace}                                      % Custom line spacing

% Figure and table packages
\usepackage{float}                                         % Figure floats
\usepackage[position=top]{subfig}                          % Subfigures
\usepackage{multirow}                                      % Merged rows in tables
\usepackage{dcolumn}                                       % Custom table delimiters
    \newcolumntype{d}[1]{D{.}{.}{#1}}
    \newcolumntype{m}[1]{D{.}{\$}{#1}}
    \newcommand\hd[1]{\multicolumn{1}{c}{#1}}
\usepackage[labelfont=bf, font=normalsize]{caption}        % Bold captions
    \captionsetup{format=hang}
\usepackage{array}                                         % Reveal table by columns
\usepackage{animate}                                       % Animated figures, GIF-like appearance
\usepackage{media9}                                        % For \mediabutton

% Graphics packages
\usepackage{tikz}                                          % TikZ drawings
\usepackage{pgfplots}                                      % PGFPlots plots
    \pgfplotsset{                                          % PGFPlots options
        cartesian/.style={                                 % Declaring Cartesian plane style
            % Axis alignment
            axis x line=middle,
            axis y line=middle,
            % Axis labels alignment
            every axis x label/.style={at={(current axis.right of origin)},anchor=north west},
            every axis y label/.style={at={(current axis.above origin)},anchor=north east},
            % Legend alignment
            every axis legend/.append style={legend pos=outer north east}
        },
        % Version declaration for compatibility
        compat=1.11
    }
\usetikzlibrary{calc}
    % Node setup for extensive form games
    \tikzset{
        % Two node styles for game trees: solid and hollow
        solid node/.style={circle,draw,inner sep=1.5,fill=black},
        hollow node/.style={circle,draw,inner sep=1.5},
        % Solution nodes
        solidsol node/.style={blue,circle,draw,inner sep=1.5,fill=blue},
        hollowsol node/.style={blue,circle,draw,inner sep=1.5}
    }

% Math packages
\usepackage{amsmath}                                       % AMS math package
\usepackage{amssymb}                                       % Math symbols
\usepackage{amsthm}                                        % Custom theorem environments
\usepackage{units}                                         % Numerical fractions
\usepackage{centernot}                                     % Logical negation in the middle of characters; e.g. not iff
\usepackage{dsfont}                                        % For indicator function: \(\mathds{1}\)

\renewcommand{\qedsymbol}{$\blacksquare$}                  % QED

% Mathematical operators
\DeclareMathOperator{\E}{\mathbb{E}}                       % Expected value
\DeclareMathOperator{\Emax}{\E\!\max}                      % Emax
\DeclareMathOperator{\var}{var}                            % Variance
\DeclareMathOperator{\cov}{cov}                            % Covariance
\DeclareMathOperator{\corr}{corr}                          % Correlation
\DeclareMathOperator{\avar}{avar}                          % Asymptotic variance
\DeclareMathOperator*{\plim}{plim}                         % Probability limit
\DeclareMathOperator{\lag}{lag}                            % Lag operator
\DeclareMathOperator{\rank}{rank}                          % Rank
\DeclareMathOperator{\tr}{tr}                              % Trace
\DeclareMathOperator{\diag}{diag}                          % Diagonal
\DeclareMathOperator{\I}{I}                                % Identity matrix

\providecommand{\abs}[1]{\left\lvert#1\right\rvert}        % Absolute value
\providecommand{\norm}[1]{\left\lVert#1\right\rVert}       % Norm
\providecommand{\ip}[1]{\left\langle#1\right\rangle}       % Inner product
\providecommand{\csp}{\overline{\mathrm{sp}}}              % Closed span
\providecommand{\pto}{\overset{p}{\to}}                    % Convergence in probability
\providecommand{\dto}{\overset{d}{\to}}                    % Convergence in distribution

% Additional commands
\newcommand\myText[1]{\text{\scriptsize\tabular[t]{@{}c@{}}#1\endtabular}}


% File paths
\newcommand{\figpath}{""}
\newcommand{\tabpath}{""}


\title{Topics in Applied Econometrics}
\subtitle[]{Regression, IV in the potential outcomes framework}
% Single author
\author[Attila Gyetvai]{Attila Gyetvai}
\institute[]{Duke Economics \\ Summer 2020}
% % Multiple authors
% \author[Gyetvai, Zhu]{Attila Gyetvai\inst{1} \and Maria Zhu\inst{1}}
% \institute[Duke] % 
% {
%  \inst{1} Department of Economics \\ Duke University \and
% }
\date{}


\begin{document}

{
\setbeamertemplate{headline}{}            % Empty header
\setbeamertemplate{footline}{}            % Empty footer  
\begin{frame}
  \titlepage
\end{frame}
}
\addtocounter{framenumber}{-1}

% ----------------------------------------------- % 
\begin{frame}[c]{How to tackle an empirical project}
  \begin{enumerate}
    \addtolength{\itemsep}{0.5\baselineskip}
    \item \textcolor{lightgray}{What causal effects are we interested in?}
    \item \textcolor{lightgray}{What ideal experiment would capture this effect?}
    \item \textcolor{lightgray}{What is our identification strategy?}
    \item What is our mode of statistical inference?
  \end{enumerate}
\end{frame}

\begin{frame}{Regression primer}
  \begin{itemize}
    \item Linear relationship between outcome and confounding factors
    \begin{align*}
      Y_i = \alpha + \beta X_i + u_i
    \end{align*}
    \item ``Best'' linear relationship fits the data best
    \item Ordinary least squares (OLS)
    \item Assumptions
    \begin{enumerate}
      \item Correct specification: \(\E(u_i \,|\, X_i) = 0\)
      \item Random sample: \((X_i, Y_i)\) are i.i.d.
      \item No extreme outliers: \(0 < \E(X_i^4) < +\infty\), \(0 < \E(u_i^4) < +\infty\)
    \end{enumerate}
    \item Under these assumptions, OLS estimates are unbiased, consistent, and \\ asymptotically normally distributed
  \end{itemize}
\end{frame}

\begin{frame}{Regression and potential outcomes}
  \begin{itemize}
    \item Treatment status as a confounder
    \item No compliance issues
    \begin{align*}
      Y_i &= \alpha + \beta D_i + u_i \\
      &= \begin{cases}
           \alpha + u_i & \text{ if } D_i=0 \\
           \alpha + \beta + u_i & \text{ if } D_i=1
         \end{cases}
      \intertext{Therefore}
      \E(Y_i \,|\, D_i=1) &= \alpha + \beta \qquad \E(Y_i \,|\, D_i=0) = \alpha
    \end{align*}
    \item If treatment status is random, ATE is \(\beta\)
  \end{itemize}
\end{frame}

\begin{frame}{Noncompliance}
  \begin{itemize}
    \item If treatment status is not random but assignment is, we still can identify LATE
    \item Recall: \(LATE = \E(Y_{1i} - Y_{0i} \,|\, D_{1i}=1, D_{0i}=0\)
    \item Let's look at compliance types
    \begin{itemize}
      \item \(\pi_C\) compliers (C): \quad\,\, \(D_{1i}=1\), \(D_{0i}=0\)
      \item \(\pi_A\) always-takers (A): \(D_{1i}=1\), \(D_{0i}=1\)
      \item \(\pi_N\) never-takers(N): \enskip \(D_{1i}=0\), \(D_{0i}=0\)
      \item No defiers
    \end{itemize}
    \item Consider the regression
    \begin{align*}
      Y_i = \alpha + \beta Z_i + u_i \pause \quad \leadsto \quad \beta = \E(Y_i \,|\, Z_i=1) - \E(Y_i \,|\, Z_i=0)
    \end{align*}
  \end{itemize}
\end{frame}

\begin{frame}{Noncompliance (cont'd)}
  \begin{itemize}
    \item Consider the regression
    \begin{align*}
      Y_i &= \alpha + \beta Z_i + u_i \pause \quad \leadsto \quad \beta = \E(Y_i \,|\, Z_i=1) - \E(Y_i \,|\, Z_i=0) \\
      \E(Y_i \,|\, Z_i=1) &= \Pr(C) \cdot \E(Y_i \,|\, Z_i=1, C) \\ &\quad+ \Pr(A) \cdot \E(Y_i \,|\, Z_i=1, A) \\&\qquad + \Pr(N) \cdot \E(Y_i \,|\, Z_i=1, N) \\
      &= \pi_C \, \E(Y_{1i} \,|\, C) + \pi_A \, \E(Y_{1i} \,|\, A) + \pi_N \, \E(Y_{0i} \,|\, N) \\
      \E(Y_i \,|\, Z_i=0) &= \Pr(C) \cdot \E(Y_i \,|\, Z_i=0, C) \\ &\quad+ \Pr(A) \cdot \E(Y_i \,|\, Z_i=0, A) \\&\qquad + \Pr(N) \cdot \E(Y_i \,|\, Z_i=0, N) \\
      &= \pi_C \, \E(Y_{0i} \,|\, C) + \pi_A \, \E(Y_{1i} \,|\, A) + \pi_N \, \E(Y_{0i} \,|\, N) \\
      \beta &= \pi_C \, \E(Y_{1i} - Y_{0i} \,|\, C)
    \end{align*}
  \end{itemize}
\end{frame}

\begin{frame}{Noncompliance (cont'd)}
  \begin{itemize}
    \item How can we get \(\pi_C\)?
    \item Consider the regression
    \begin{align*}
      D_i &= \gamma + \delta Z_i + u_i \pause \quad \leadsto \quad \delta = \E(D_i \,|\, Z_i=1) - \E(D_i \,|\, Z_i=0) = \pi_C
    \end{align*}
    \item Therefore we get LATE as
    \begin{align*}
      LATE = \E(Y_{1i} - Y_{0i} \,|\, C) = \frac{\beta}{\delta} = \frac{\E(Y_i \,|\, Z_i=1) - \E(Y_i \,|\, Z_i=0)}{\E(D_i \,|\, Z_i=1) - \E(D_i \,|\, Z_i=0)}
    \end{align*}
    \item This is the Wald estimator, i.e., IV with binary instrument
  \end{itemize}
\end{frame}

\begin{frame}{LATE as IV}
  \begin{itemize}
    \item IV: instrumental variables regression
    \item \(Y_i = \alpha + \beta D_i + u_i\) with endogenous regressor \(D\) (\(\E(u_i \,|\, D_i) \neq 0\))
    \item \(D_i = \gamma + \delta Z_i + v_i\) with exogenous instrument \(Z\) (\(\E(v_i \,|\, Z_i) = 0\))
    \item Two-stage estimation (2SLS)
    \begin{enumerate}
      \item Regress \(D\) on \(Z\), obtain predicted \(\hat{D}\)
      \item Regress \(Y\) on \(\hat{D}\)
    \end{enumerate}
    \item IV identifies LATE
  \end{itemize}
\end{frame}

\begin{frame}{Regression, IV in practice}
  \begin{itemize}
    \item Canned procedures in statistical software 
    \item Stata:
    \begin{itemize}
      \item \texttt{reg Y D, r}
      \item \texttt{ivregress 2sls Y (D=Z), vce(r)}
    \end{itemize}
    \item R:
    \begin{itemize}
      \item \texttt{lm(Y \(\sim\) D, data)} \qquad \qquad (plus extra stuff for robust errors)
      \item \texttt{ivreg(Y \(\sim\) D | Z, data)}
    \end{itemize}
  \end{itemize}
\end{frame}

\end{document}
 