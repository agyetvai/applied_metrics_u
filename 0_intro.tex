\documentclass[aspectratio=169,compress,handout,t,xcolor=table]{beamer}

%! Mandatory packages
\usepackage[utf8]{inputenc}                                %! Character encoding
\usepackage[T1]{fontenc}                                   %! Font encoding
\usepackage[english]{babel}                                %! Language setting
\usepackage{graphicx}

%%%% Frame design
\usetheme{Boadilla}                                        % Design base
\usecolortheme{default}                                    % Color structure base

% Colors
\definecolor{MyBackground}{HTML}{FFFAEE}                   % Cream color with dark text works well for dyslexic readers: FFFAEE
\definecolor{MyBackground}{HTML}{FFFFFF}                   % White for online slides
\setbeamercolor{background canvas}{bg=MyBackground}
\definecolor{MyStructure}{HTML}{005471}
\setbeamercolor{title}{fg=MyStructure}
\setbeamercolor{frametitle}{fg=MyStructure}
\setbeamercolor{structure}{fg=MyStructure}

% Fonts
\usepackage{tgheros}                                       % TG Heros font package
\beamertemplatenavigationsymbolsempty                      % Disable navigation bar

% Enumerate / itemize design
\setbeamertemplate{enumerate items}[circle]
\setbeamertemplate{itemize items}[circle]
\setbeamertemplate{itemize subitem}[triangle]
\setbeamertemplate{section in toc}[sections numbered]
\usepackage{xpatch}                                        % Global itemize/enumerate list separation
\xpatchcmd{\itemize}
  {\def\makelabel}
  {\ifnum\@itemdepth=1\relax
     \setlength\itemsep{2ex}% separation for first level
   \else
     \ifnum\@itemdepth=2\relax
       \setlength\itemsep{1ex}% separation for second level
     \else
       \ifnum\@itemdepth=3\relax
         \setlength\itemsep{0.5ex}% separation for third level
   \fi\fi\fi\def\makelabel
  }
{}
{}


% Footer
\setbeamertemplate{footline}
{
 \hbox{%
  \begin{beamercolorbox}[wd=.2\paperwidth,ht=2.6ex,dp=1ex,center]{section in foot}%
    \usebeamerfont{section in foot}\insertshortauthor
  \end{beamercolorbox}%
  \begin{beamercolorbox}[wd=.75\paperwidth,ht=2.6ex,dp=1ex,center]{section in foot}%
    \usebeamerfont{section in foot}\insertsubtitle
  \end{beamercolorbox}%
  \begin{beamercolorbox}[wd=.05\paperwidth,ht=2.6ex,dp=1ex,center]{subsection in foot}%
    \usebeamerfont{section in foot} \insertframenumber/\inserttotalframenumber
  \end{beamercolorbox}}%

  \vskip0pt%
}
\setbeamertemplate{frametitle}[default][left]              % Push frame title to left
\setbeamertemplate{navigation symbols}{}                   % Remove navigation symbols

\renewcommand\textbullet{\ensuremath{\bullet}}             % Opress textbullet font warning
\usepackage{appendixnumberbeamer}                          % Restart slide numbering for backup slides (\appendix)

% Roadmap
\AtBeginSection[]
{
	\ifnum \value{framenumber}>1
   \begin{frame}[plain]
       \frametitle{Roadmap}
       \tableofcontents[currentsection]
   \end{frame}
   \addtocounter{framenumber}{-1}
  \else
   \fi
}

% Stable preamble with path
% Text packages
\usepackage{enumerate}                                     % Custom enumeration lists
\usepackage{footmisc}                                      % Stable footnotes in headings
\usepackage{natbib}                                        % Bibliography
\usepackage{hyperref}                                      % Hyperreferences
    \hypersetup{
      % colorlinks,
      % citecolor=black,
      % filecolor=black,
      % linkcolor=black,
      urlcolor=MyStructure
    }
\usepackage{multicol}                                      % Multiple columns in text
\usepackage[nointegrals]{wasysym}                          % WASY2 symbols for contradiction \lightning
\usepackage{algorithm}                                     % Pseudocode packages
\usepackage{algpseudocode}
\usepackage{setspace}                                      % Custom line spacing

% Figure and table packages
\usepackage{float}                                         % Figure floats
\usepackage[position=top]{subfig}                          % Subfigures
\usepackage{multirow}                                      % Merged rows in tables
\usepackage{dcolumn}                                       % Custom table delimiters
    \newcolumntype{d}[1]{D{.}{.}{#1}}
    \newcolumntype{m}[1]{D{.}{\$}{#1}}
    \newcommand\hd[1]{\multicolumn{1}{c}{#1}}
\usepackage[labelfont=bf, font=normalsize]{caption}        % Bold captions
    \captionsetup{format=hang}
\usepackage{array}                                         % Reveal table by columns
\usepackage{animate}                                       % Animated figures, GIF-like appearance
\usepackage{media9}                                        % For \mediabutton

% Graphics packages
\usepackage{tikz}                                          % TikZ drawings
\usepackage{pgfplots}                                      % PGFPlots plots
    \pgfplotsset{                                          % PGFPlots options
        cartesian/.style={                                 % Declaring Cartesian plane style
            % Axis alignment
            axis x line=middle,
            axis y line=middle,
            % Axis labels alignment
            every axis x label/.style={at={(current axis.right of origin)},anchor=north west},
            every axis y label/.style={at={(current axis.above origin)},anchor=north east},
            % Legend alignment
            every axis legend/.append style={legend pos=outer north east}
        },
        % Version declaration for compatibility
        compat=1.11
    }
\usetikzlibrary{calc}
    % Node setup for extensive form games
    \tikzset{
        % Two node styles for game trees: solid and hollow
        solid node/.style={circle,draw,inner sep=1.5,fill=black},
        hollow node/.style={circle,draw,inner sep=1.5},
        % Solution nodes
        solidsol node/.style={blue,circle,draw,inner sep=1.5,fill=blue},
        hollowsol node/.style={blue,circle,draw,inner sep=1.5}
    }

% Math packages
\usepackage{amsmath}                                       % AMS math package
\usepackage{amssymb}                                       % Math symbols
\usepackage{amsthm}                                        % Custom theorem environments
\usepackage{units}                                         % Numerical fractions
\usepackage{centernot}                                     % Logical negation in the middle of characters; e.g. not iff
\usepackage{dsfont}                                        % For indicator function: \(\mathds{1}\)

\renewcommand{\qedsymbol}{$\blacksquare$}                  % QED

% Mathematical operators
\DeclareMathOperator{\E}{\mathbb{E}}                       % Expected value
\DeclareMathOperator{\Emax}{\E\!\max}                      % Emax
\DeclareMathOperator{\var}{var}                            % Variance
\DeclareMathOperator{\cov}{cov}                            % Covariance
\DeclareMathOperator{\corr}{corr}                          % Correlation
\DeclareMathOperator{\avar}{avar}                          % Asymptotic variance
\DeclareMathOperator*{\plim}{plim}                         % Probability limit
\DeclareMathOperator{\lag}{lag}                            % Lag operator
\DeclareMathOperator{\rank}{rank}                          % Rank
\DeclareMathOperator{\tr}{tr}                              % Trace
\DeclareMathOperator{\diag}{diag}                          % Diagonal
\DeclareMathOperator{\I}{I}                                % Identity matrix

\providecommand{\abs}[1]{\left\lvert#1\right\rvert}        % Absolute value
\providecommand{\norm}[1]{\left\lVert#1\right\rVert}       % Norm
\providecommand{\ip}[1]{\left\langle#1\right\rangle}       % Inner product
\providecommand{\csp}{\overline{\mathrm{sp}}}              % Closed span
\providecommand{\pto}{\overset{p}{\to}}                    % Convergence in probability
\providecommand{\dto}{\overset{d}{\to}}                    % Convergence in distribution

% Additional commands
\newcommand\myText[1]{\text{\scriptsize\tabular[t]{@{}c@{}}#1\endtabular}}


% File paths
\newcommand{\figpath}{""}
\newcommand{\tabpath}{""}


\title{Topics in Applied Econometrics}
\subtitle[]{Challenges to identify causal effects}
% Single author
\author[Attila Gyetvai]{Attila Gyetvai}
\institute[]{Duke Economics \\ Summer 2020}
% % Multiple authors
% \author[Gyetvai, Zhu]{Attila Gyetvai\inst{1} \and Maria Zhu\inst{1}}
% \institute[Duke] % 
% {
%  \inst{1} Department of Economics \\ Duke University \and
% }
\date{}


\begin{document}

{
\setbeamertemplate{headline}{}            % Empty header
\setbeamertemplate{footline}{}            % Empty footer  
\begin{frame}
  \titlepage
\end{frame}
}
\addtocounter{framenumber}{-1}

% ----------------------------------------------- % 
\begin{frame}[c]{How to tackle an empirical project}
  \begin{enumerate}
    \addtolength{\itemsep}{0.5\baselineskip}
    \item What causal effects are we interested in?
    \item What ideal experiment would capture this effect?
    \item What is our identification strategy?
    \item What is our mode of statistical inference?
  \end{enumerate}
\end{frame}

\begin{frame}{Establishing some language}
  \begin{itemize}
    \item ``Treatments''
    \begin{itemize}
      \item Terminology and some tools come from the medical literature
      \item Any causal effect can be thought of as the effect of some ``treatment''
    \end{itemize}
    \item We are interested in the outcome \(Y\)
    \item Consider the effect of the treatment on a potential participant \(i\)
    \item Two states of the world:
    \begin{itemize}
      \item \(Y_{1i}\) if \(i\) is treated
      \item \(Y_{0i}\) if \(i\) is not treated
    \end{itemize}
    \item \(Y_{1i} - Y_{0i}\) is the effect of the treatment on \(i\)
  \end{itemize}
\end{frame}

\begin{frame}{Counterfactual approach}
  \begin{itemize}
    \item We can be in only one state of the world
    \item \(i\) is either treated or not
    \begin{itemize}
      \item If \(i\) is treated, we can observe \(Y_{1i}\) but not \(Y_{0i}\)
      \item If \(i\) is not treated, we can observe \(Y_{0i}\) but not \(Y_{1i}\)
    \end{itemize}
    \item The unobserved outcome is the \emph{counterfactual} outcome
  \end{itemize}
\end{frame}

\begin{frame}[c]
  \begin{block}{Fundamental problem of causal inference}
    \(Y_{1i} - Y_{0i}\) cannot be observed
  \end{block}

  \begin{itemize}
    \item We can only see the actual but not the counterfactual outcome
    \item Therefore the effect of the treatment is fundamentally unobservable
    \item However, not all hope is lost!
  \end{itemize}
\end{frame}

\begin{frame}{Measuring the counterfactual}
  \begin{itemize}
    \item Observation vs.\ measurement
    \begin{itemize}
      \item We cannot observe the counterfactual outcome
      \item \ldots but we can measure it under certain assumptions
    \end{itemize}
    \item \emph{Example:} \(Y_{1i} = Y_j\) when \(j\) is treated and \(i\) is not
    \begin{itemize}
      \item This assumption is simple but very restrictive
    \end{itemize}
    \item \emph{Question:} What is the least restrictive set of assumptions under which \\ we can identify the effect of the treatment?
    \item During this course, we will cover various sets of such assumptions
  \end{itemize}
\end{frame}

\begin{frame}{The potential outcomes framework}
  \begin{itemize}
    \item Potential outcome: what the counterfactual outcome could be
    \item Treatment variable: \(D_i = 1\) if \(i\) is treated \(\big/\) 0 if \(i\) is not treated
    \item Observed outcome:
    \begin{align*}
      Y_i &= \begin{cases}
        Y_{1i} & \text{ if } D_i = 1 \\
        Y_{0i} & \text{ if } D_i = 0
      \end{cases}
    \end{align*}
  \end{itemize}
\end{frame}

\begin{frame}{Starting to think about treatment effects}
  \begin{itemize}
    \item The effect of the treatment is \(Y_{1i} - Y_{0i}\)
    \item How can we say something meaningful about this effect?
    \item We need to aggregate outcomes within groups, i.e., take expectations
    \item \emph{Example:} the average treatment effect (ATE)
    \begin{align*}
      ATE &= \E(Y_{1i} - Y_{0i}) = \E(Y_{1i}) - \E(Y_{0i})
    \end{align*}
  \end{itemize}
\end{frame}

\begin{frame}{Average treatment effect}
  \begin{align*}
    ATE &= \E(Y_{1i} - Y_{0i}) = \E(Y_{1i}) - \E(Y_{0i})
  \end{align*}
  \begin{itemize}
    \item Recall the fundamental problem: we can only observe the actual outcomes
    \item How can we measure the ATE from these observations?
    \item We need to make assumptions about the counterfactual outcomes \\ \(\E (Y_{1i} \, | D_i=0)\) and \(\E (Y_{0i} \, | D_i=1)\)
    \item E.g., assume that the potential outcomes in the treated and untreated groups \\ are the same
    \item When is this assumption reasonable?
  \end{itemize}
\end{frame}

\begin{frame}{Challenges arise}
  \begin{itemize}
    \item Potential outcomes are not the same
    \begin{itemize}
      \item Units ``select into treatment'' (choose treatment status)
      \item Selection on observable vs.\ unobservable characteristics
    \end{itemize}
    \item Treated and untreated groups are not well-defined
    \begin{itemize}
      \item Some units opt out of the treatment
      \item Some units fight their way into treatment, even though they are not supposed to
    \end{itemize}
    \item The treatment effect is different for various subgroups
    \begin{itemize}
      \item The effect is larger for units with certain characteristics
      \item The effect is larger for those who are ``just treated''
    \end{itemize}
    \item And so on\ldots
  \end{itemize}
\end{frame}

\begin{frame}{Evaluating the treatment effect}
  \begin{itemize}
    \item Suppose we successfully measure the treatment effect. What does it tell us?
    \item In social sciences, we are mostly interested in generalizing the findings from a certain program to other settings
    \item Internal validity: Does the analysis show the effect of the particular program?
    % \begin{itemize}
    %   \item 
    % \end{itemize}
    \item External validity: Can the effects be generalized to other programs?
    % \begin{itemize}
    %   \item Can the effects be generalized to other programs?
    % \end{itemize}
    \item Internal vs.\ external validity:
    \begin{itemize}
      \item One can argue that only internally valid results should be generalized
      \item \ldots but the more tailored the analysis to a certain program, the less it is generalizable
    \end{itemize}
  \end{itemize}
\end{frame}

\begin{frame}{Looking forward}
  \begin{itemize}    
    \item \textbf{We} will cover the cornerstones of a successful empirical project
    \begin{enumerate}
      \addtolength{\itemsep}{0.5\baselineskip}
      \item What causal effects are we interested in?
      \item What ideal experiment would capture this effect?
      \item What is our identification strategy?
      \item What is our mode of statistical inference?
    \end{enumerate}
    \item \textbf{You} will tackle an empirical project along these principles
    \begin{itemize}
      \item Pose an empirical question
      \item Design a project that seeks to answer the question
      \item Describe the ideal experiment
      \item Simulate data from a non-ideal experiment
      \item Evaluate the treatment effect
    \end{itemize}
  \end{itemize}  
\end{frame}

\begin{frame}{Empirical project}
  \begin{itemize}
    \item You will work in groups
    \item Groups will be formed after the add/drop date
    \item Groups will regularly update me on their progress in Zoom meetings
    \item At the end of the course, each group will present their project
  \end{itemize}
\end{frame}




\end{document}
 