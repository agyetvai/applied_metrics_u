\documentclass[aspectratio=169,compress,handout,t,xcolor=table]{beamer}

%! Mandatory packages
\usepackage[utf8]{inputenc}                                %! Character encoding
\usepackage[T1]{fontenc}                                   %! Font encoding
\usepackage[english]{babel}                                %! Language setting
\usepackage{graphicx}

%%%% Frame design
\usetheme{Boadilla}                                        % Design base
\usecolortheme{default}                                    % Color structure base

% Colors
\definecolor{MyBackground}{HTML}{FFFAEE}                   % Cream color with dark text works well for dyslexic readers: FFFAEE
\definecolor{MyBackground}{HTML}{FFFFFF}                   % White for online slides
\setbeamercolor{background canvas}{bg=MyBackground}
\definecolor{MyStructure}{HTML}{005471}
\setbeamercolor{title}{fg=MyStructure}
\setbeamercolor{frametitle}{fg=MyStructure}
\setbeamercolor{structure}{fg=MyStructure}

% Fonts
\usepackage{tgheros}                                       % TG Heros font package
\beamertemplatenavigationsymbolsempty                      % Disable navigation bar

% Enumerate / itemize design
\setbeamertemplate{enumerate items}[circle]
\setbeamertemplate{itemize items}[circle]
\setbeamertemplate{itemize subitem}[triangle]
\setbeamertemplate{section in toc}[sections numbered]
\usepackage{xpatch}                                        % Global itemize/enumerate list separation
\xpatchcmd{\itemize}
  {\def\makelabel}
  {\ifnum\@itemdepth=1\relax
     \setlength\itemsep{2ex}% separation for first level
   \else
     \ifnum\@itemdepth=2\relax
       \setlength\itemsep{1ex}% separation for second level
     \else
       \ifnum\@itemdepth=3\relax
         \setlength\itemsep{0.5ex}% separation for third level
   \fi\fi\fi\def\makelabel
  }
{}
{}


% Footer
\setbeamertemplate{footline}
{
 \hbox{%
  \begin{beamercolorbox}[wd=.2\paperwidth,ht=2.6ex,dp=1ex,center]{section in foot}%
    \usebeamerfont{section in foot}\insertshortauthor
  \end{beamercolorbox}%
  \begin{beamercolorbox}[wd=.75\paperwidth,ht=2.6ex,dp=1ex,center]{section in foot}%
    \usebeamerfont{section in foot}\insertsubtitle
  \end{beamercolorbox}%
  \begin{beamercolorbox}[wd=.05\paperwidth,ht=2.6ex,dp=1ex,center]{subsection in foot}%
    \usebeamerfont{section in foot} \insertframenumber/\inserttotalframenumber
  \end{beamercolorbox}}%

  \vskip0pt%
}
\setbeamertemplate{frametitle}[default][left]              % Push frame title to left
\setbeamertemplate{navigation symbols}{}                   % Remove navigation symbols

\renewcommand\textbullet{\ensuremath{\bullet}}             % Opress textbullet font warning
\usepackage{appendixnumberbeamer}                          % Restart slide numbering for backup slides (\appendix)

% Roadmap
\AtBeginSection[]
{
	\ifnum \value{framenumber}>0
   \begin{frame}[plain]
       \frametitle{Plan for today}
       \tableofcontents[currentsection]
   \end{frame}
   \addtocounter{framenumber}{-1}
  \else
   \fi
}

% Stable preamble with path
% Text packages
\usepackage{enumerate}                                     % Custom enumeration lists
\usepackage{footmisc}                                      % Stable footnotes in headings
\usepackage{natbib}                                        % Bibliography
\usepackage{hyperref}                                      % Hyperreferences
    \hypersetup{
      % colorlinks,
      % citecolor=black,
      % filecolor=black,
      % linkcolor=black,
      urlcolor=MyStructure
    }
\usepackage{multicol}                                      % Multiple columns in text
\usepackage[nointegrals]{wasysym}                          % WASY2 symbols for contradiction \lightning
\usepackage{algorithm}                                     % Pseudocode packages
\usepackage{algpseudocode}
\usepackage{setspace}                                      % Custom line spacing

% Figure and table packages
\usepackage{float}                                         % Figure floats
\usepackage[position=top]{subfig}                          % Subfigures
\usepackage{multirow}                                      % Merged rows in tables
\usepackage{dcolumn}                                       % Custom table delimiters
    \newcolumntype{d}[1]{D{.}{.}{#1}}
    \newcolumntype{m}[1]{D{.}{\$}{#1}}
    \newcommand\hd[1]{\multicolumn{1}{c}{#1}}
\usepackage[labelfont=bf, font=normalsize]{caption}        % Bold captions
    \captionsetup{format=hang}
\usepackage{array}                                         % Reveal table by columns
\usepackage{animate}                                       % Animated figures, GIF-like appearance
\usepackage{media9}                                        % For \mediabutton

% Graphics packages
\usepackage{tikz}                                          % TikZ drawings
\usepackage{pgfplots}                                      % PGFPlots plots
    \pgfplotsset{                                          % PGFPlots options
        cartesian/.style={                                 % Declaring Cartesian plane style
            % Axis alignment
            axis x line=middle,
            axis y line=middle,
            % Axis labels alignment
            every axis x label/.style={at={(current axis.right of origin)},anchor=north west},
            every axis y label/.style={at={(current axis.above origin)},anchor=north east},
            % Legend alignment
            every axis legend/.append style={legend pos=outer north east}
        },
        % Version declaration for compatibility
        compat=1.11
    }
\usetikzlibrary{calc}
    % Node setup for extensive form games
    \tikzset{
        % Two node styles for game trees: solid and hollow
        solid node/.style={circle,draw,inner sep=1.5,fill=black},
        hollow node/.style={circle,draw,inner sep=1.5},
        % Solution nodes
        solidsol node/.style={blue,circle,draw,inner sep=1.5,fill=blue},
        hollowsol node/.style={blue,circle,draw,inner sep=1.5}
    }

% Math packages
\usepackage{amsmath}                                       % AMS math package
\usepackage{amssymb}                                       % Math symbols
\usepackage{amsthm}                                        % Custom theorem environments
\usepackage{units}                                         % Numerical fractions
\usepackage{centernot}                                     % Logical negation in the middle of characters; e.g. not iff
\usepackage{dsfont}                                        % For indicator function: \(\mathds{1}\)

\renewcommand{\qedsymbol}{$\blacksquare$}                  % QED

% Mathematical operators
\DeclareMathOperator{\E}{\mathbb{E}}                       % Expected value
\DeclareMathOperator{\Emax}{\E\!\max}                      % Emax
\DeclareMathOperator{\var}{var}                            % Variance
\DeclareMathOperator{\cov}{cov}                            % Covariance
\DeclareMathOperator{\corr}{corr}                          % Correlation
\DeclareMathOperator{\avar}{avar}                          % Asymptotic variance
\DeclareMathOperator*{\plim}{plim}                         % Probability limit
\DeclareMathOperator{\lag}{lag}                            % Lag operator
\DeclareMathOperator{\rank}{rank}                          % Rank
\DeclareMathOperator{\tr}{tr}                              % Trace
\DeclareMathOperator{\diag}{diag}                          % Diagonal
\DeclareMathOperator{\I}{I}                                % Identity matrix

\providecommand{\abs}[1]{\left\lvert#1\right\rvert}        % Absolute value
\providecommand{\norm}[1]{\left\lVert#1\right\rVert}       % Norm
\providecommand{\ip}[1]{\left\langle#1\right\rangle}       % Inner product
\providecommand{\csp}{\overline{\mathrm{sp}}}              % Closed span
\providecommand{\pto}{\overset{p}{\to}}                    % Convergence in probability
\providecommand{\dto}{\overset{d}{\to}}                    % Convergence in distribution
\providecommand{\indep}{\perp \!\!\!\! \perp}              % Independent

% Additional commands
\newcommand\myText[1]{\text{\scriptsize\tabular[t]{@{}c@{}}#1\endtabular}}


% File paths
\newcommand{\figpath}{""}
\newcommand{\tabpath}{""}


\title{Topics in Applied Econometrics}
\subtitle[]{Regression discontinuity}
% Single author
\author[Attila Gyetvai]{Attila Gyetvai}
\institute[]{Duke Economics \\ Summer 2020}
% % Multiple authors
% \author[Gyetvai, Zhu]{Attila Gyetvai\inst{1} \and Maria Zhu\inst{1}}
% \institute[Duke] % 
% {
%  \inst{1} Department of Economics \\ Duke University \and
% }
\date{}


\begin{document}

{
\setbeamertemplate{headline}{}            % Empty header
\setbeamertemplate{footline}{}            % Empty footer  
\begin{frame}
  \titlepage
\end{frame}
}
\addtocounter{framenumber}{-1}

% ----------------------------------------------- % 
\begin{frame}[c]{How to tackle an empirical project}
  \begin{enumerate}
    \addtolength{\itemsep}{0.5\baselineskip}
    \item \textcolor{lightgray}{What causal effects are we interested in?}
    \item \textcolor{lightgray}{What ideal experiment would capture this effect?}
    \item What is our identification strategy?
    \item \textcolor{lightgray}{What is our mode of statistical inference?}
  \end{enumerate}
\end{frame}

\begin{frame}{Empirical setting into research design}
  \begin{itemize}
    \item So far, we have considered the two extremes of randomization
    \begin{itemize}
      \item Perfect randomization: RCT
      \item No randomization: matching
    \end{itemize}
    \item Sometimes the truth lies in between
    \item \emph{Example:} unemployment insurance
    \begin{itemize}
      \item Some baseline level of UI benefits
      \item For some unemployed in certain regions, additional UI
      \item What can we learn from comparing them to those in neighboring regions? 
    \end{itemize}
  \end{itemize}
\end{frame}

\begin{frame}{Treatment thresholds}
  \begin{itemize}
    \item Treatment status depends on some eligibility threshold
    \item ``Just treated'' units are comparable to ``just untreated'' units
    \item Assignment to treatment at the threshold is as good as random
    \item How reasonable is this assumption?
    \begin{itemize}
      \item Age cutoff: born on June 30 vs.\ July 1
      \item Area border: east vs.\ west side of a street
    \end{itemize}
    \item How strict is the cutoff?
    \begin{itemize}
      \item Sharp RD: no one is treated on one side, everyone is treated on the other
      \item Fuzzy RD: some may be treated on both sides, but the fraction of treated units jumps at the threshold
    \end{itemize}
  \end{itemize}
\end{frame}

\begin{frame}{Treatment effects with sharp RD}
  \begin{itemize}
    \item Does a sharp RD identify ATE?
    \item Yes\ldots right around the threshold, if compliance is perfect
    \item Hard to argue that the effect can be extrapolated to further away from the threshold
    \item \emph{Example:} UI benefits in Austria \citep*[][JOE]{Lalive2008}
    \begin{itemize}
      \item Under 50: up to 39 weeks of UI
      \item 50+: 52 weeks \(\leadsto\) 209 weeks in certain regions
    \end{itemize}
  \end{itemize}
\end{frame}

\begin{frame}{Treatment effects with sharp RD (cont'd)}
  \begin{center}
    \includegraphics[height=0.8\textheight]{fig2.png} \\
    {\tiny \emph{Source:} \citet*{Lalive2008}}
  \end{center}
\end{frame}
\addtocounter{framenumber}{-1}

\begin{frame}{Treatment effects with sharp RD (cont'd)}
  \begin{center}
    \includegraphics[height=0.8\textheight]{fig7.png} \\
    {\tiny \emph{Source:} \citet*{Lalive2008}}
  \end{center}
\end{frame}
\addtocounter{framenumber}{-1}

\begin{frame}{Treatment effects with sharp RD (cont'd)}
  \begin{center}
    \includegraphics[height=0.8\textheight]{fig3.png} \\
    {\tiny \emph{Source:} \citet*{Lalive2008}}
  \end{center}
\end{frame}
\addtocounter{framenumber}{-1}

\begin{frame}{Treatment effects with sharp RD (cont'd)}
  \begin{center}
    \includegraphics[height=0.8\textheight]{fig8.png} \\
    {\tiny \emph{Source:} \citet*{Lalive2008}}
  \end{center}
\end{frame}

\begin{frame}{Treatment effects with fuzzy RD}
  \begin{itemize}
    \item Does a fuzzy RD identify ATE?
    \item No, not even under perfect compliance
    \item But it identifies\ldots LATE!
    \item Zoom in on the treated units on the treated side of the threshold
    \item Even harder to extrapolate results, but arguably valid at the threshold
    \item \emph{Example:} class size and test scores \citep*[][QJE]{Angrist1999}
  \end{itemize}
\end{frame}
\addtocounter{framenumber}{-1}

\begin{frame}{Treatment effects with fuzzy RD}
  \begin{center}
    \includegraphics[height=0.8\textheight]{figa.png} \\
    {\tiny \emph{Source:} \citet*{Angrist1999}}
  \end{center}
\end{frame}
\addtocounter{framenumber}{-1}

\begin{frame}{Treatment effects with fuzzy RD}
  \begin{center}
    \includegraphics[height=0.8\textheight]{figb.png} \\
    {\tiny \emph{Source:} \citet*{Angrist1999}}
  \end{center}
\end{frame}

\appendix
\begin{frame}[plain,c]
  \centerline{\Large{{\usebeamercolor[fg]{title} Additional Slides}}}
\end{frame}
\addtocounter{framenumber}{-1}
\begin{frame}[allowframebreaks]{References}
  \bibliographystyle{mychicago}
  \linespread{1}
  \begin{footnotesize}
  \bibliography{../refs}
  \end{footnotesize}
\end{frame}


\end{document}
 