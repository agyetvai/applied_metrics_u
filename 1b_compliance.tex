\documentclass[aspectratio=169,compress,handout,t,xcolor=table]{beamer}

%! Mandatory packages
\usepackage[utf8]{inputenc}                                %! Character encoding
\usepackage[T1]{fontenc}                                   %! Font encoding
\usepackage[english]{babel}                                %! Language setting
\usepackage{graphicx}

%%%% Frame design
\usetheme{Boadilla}                                        % Design base
\usecolortheme{default}                                    % Color structure base

% Colors
\definecolor{MyBackground}{HTML}{FFFAEE}                   % Cream color with dark text works well for dyslexic readers: FFFAEE
\definecolor{MyBackground}{HTML}{FFFFFF}                   % White for online slides
\setbeamercolor{background canvas}{bg=MyBackground}
\definecolor{MyStructure}{HTML}{005471}
\setbeamercolor{title}{fg=MyStructure}
\setbeamercolor{frametitle}{fg=MyStructure}
\setbeamercolor{structure}{fg=MyStructure}

% Fonts
\usepackage{tgheros}                                       % TG Heros font package
\beamertemplatenavigationsymbolsempty                      % Disable navigation bar

% Enumerate / itemize design
\setbeamertemplate{enumerate items}[circle]
\setbeamertemplate{itemize items}[circle]
\setbeamertemplate{itemize subitem}[triangle]
\setbeamertemplate{section in toc}[sections numbered]
\usepackage{xpatch}                                        % Global itemize/enumerate list separation
\xpatchcmd{\itemize}
  {\def\makelabel}
  {\ifnum\@itemdepth=1\relax
     \setlength\itemsep{2ex}% separation for first level
   \else
     \ifnum\@itemdepth=2\relax
       \setlength\itemsep{1ex}% separation for second level
     \else
       \ifnum\@itemdepth=3\relax
         \setlength\itemsep{0.5ex}% separation for third level
   \fi\fi\fi\def\makelabel
  }
{}
{}


% Footer
\setbeamertemplate{footline}
{
 \hbox{%
  \begin{beamercolorbox}[wd=.2\paperwidth,ht=2.6ex,dp=1ex,center]{section in foot}%
    \usebeamerfont{section in foot}\insertshortauthor
  \end{beamercolorbox}%
  \begin{beamercolorbox}[wd=.75\paperwidth,ht=2.6ex,dp=1ex,center]{section in foot}%
    \usebeamerfont{section in foot}\insertsubtitle
  \end{beamercolorbox}%
  \begin{beamercolorbox}[wd=.05\paperwidth,ht=2.6ex,dp=1ex,center]{subsection in foot}%
    \usebeamerfont{section in foot} \insertframenumber/\inserttotalframenumber
  \end{beamercolorbox}}%

  \vskip0pt%
}
\setbeamertemplate{frametitle}[default][left]              % Push frame title to left
\setbeamertemplate{navigation symbols}{}                   % Remove navigation symbols

\renewcommand\textbullet{\ensuremath{\bullet}}             % Opress textbullet font warning
\usepackage{appendixnumberbeamer}                          % Restart slide numbering for backup slides (\appendix)

% Roadmap
\AtBeginSection[]
{
	\ifnum \value{framenumber}>0
   \begin{frame}[plain]
       \frametitle{Plan for today}
       \tableofcontents[currentsection]
   \end{frame}
   \addtocounter{framenumber}{-1}
  \else
   \fi
}

% Stable preamble with path
% Text packages
\usepackage{enumerate}                                     % Custom enumeration lists
\usepackage{footmisc}                                      % Stable footnotes in headings
\usepackage{natbib}                                        % Bibliography
\usepackage{hyperref}                                      % Hyperreferences
    \hypersetup{
      % colorlinks,
      % citecolor=black,
      % filecolor=black,
      % linkcolor=black,
      urlcolor=MyStructure
    }
\usepackage{multicol}                                      % Multiple columns in text
\usepackage[nointegrals]{wasysym}                          % WASY2 symbols for contradiction \lightning
\usepackage{algorithm}                                     % Pseudocode packages
\usepackage{algpseudocode}
\usepackage{setspace}                                      % Custom line spacing

% Figure and table packages
\usepackage{float}                                         % Figure floats
\usepackage[position=top]{subfig}                          % Subfigures
\usepackage{multirow}                                      % Merged rows in tables
\usepackage{dcolumn}                                       % Custom table delimiters
    \newcolumntype{d}[1]{D{.}{.}{#1}}
    \newcolumntype{m}[1]{D{.}{\$}{#1}}
    \newcommand\hd[1]{\multicolumn{1}{c}{#1}}
\usepackage[labelfont=bf, font=normalsize]{caption}        % Bold captions
    \captionsetup{format=hang}
\usepackage{array}                                         % Reveal table by columns
\usepackage{animate}                                       % Animated figures, GIF-like appearance
\usepackage{media9}                                        % For \mediabutton

% Graphics packages
\usepackage{tikz}                                          % TikZ drawings
\usepackage{pgfplots}                                      % PGFPlots plots
    \pgfplotsset{                                          % PGFPlots options
        cartesian/.style={                                 % Declaring Cartesian plane style
            % Axis alignment
            axis x line=middle,
            axis y line=middle,
            % Axis labels alignment
            every axis x label/.style={at={(current axis.right of origin)},anchor=north west},
            every axis y label/.style={at={(current axis.above origin)},anchor=north east},
            % Legend alignment
            every axis legend/.append style={legend pos=outer north east}
        },
        % Version declaration for compatibility
        compat=1.11
    }
\usetikzlibrary{calc}
    % Node setup for extensive form games
    \tikzset{
        % Two node styles for game trees: solid and hollow
        solid node/.style={circle,draw,inner sep=1.5,fill=black},
        hollow node/.style={circle,draw,inner sep=1.5},
        % Solution nodes
        solidsol node/.style={blue,circle,draw,inner sep=1.5,fill=blue},
        hollowsol node/.style={blue,circle,draw,inner sep=1.5}
    }

% Math packages
\usepackage{amsmath}                                       % AMS math package
\usepackage{amssymb}                                       % Math symbols
\usepackage{amsthm}                                        % Custom theorem environments
\usepackage{units}                                         % Numerical fractions
\usepackage{centernot}                                     % Logical negation in the middle of characters; e.g. not iff
\usepackage{dsfont}                                        % For indicator function: \(\mathds{1}\)

\renewcommand{\qedsymbol}{$\blacksquare$}                  % QED

% Mathematical operators
\DeclareMathOperator{\E}{\mathbb{E}}                       % Expected value
\DeclareMathOperator{\Emax}{\E\!\max}                      % Emax
\DeclareMathOperator{\var}{var}                            % Variance
\DeclareMathOperator{\cov}{cov}                            % Covariance
\DeclareMathOperator{\corr}{corr}                          % Correlation
\DeclareMathOperator{\avar}{avar}                          % Asymptotic variance
\DeclareMathOperator*{\plim}{plim}                         % Probability limit
\DeclareMathOperator{\lag}{lag}                            % Lag operator
\DeclareMathOperator{\rank}{rank}                          % Rank
\DeclareMathOperator{\tr}{tr}                              % Trace
\DeclareMathOperator{\diag}{diag}                          % Diagonal
\DeclareMathOperator{\I}{I}                                % Identity matrix

\providecommand{\abs}[1]{\left\lvert#1\right\rvert}        % Absolute value
\providecommand{\norm}[1]{\left\lVert#1\right\rVert}       % Norm
\providecommand{\ip}[1]{\left\langle#1\right\rangle}       % Inner product
\providecommand{\csp}{\overline{\mathrm{sp}}}              % Closed span
\providecommand{\pto}{\overset{p}{\to}}                    % Convergence in probability
\providecommand{\dto}{\overset{d}{\to}}                    % Convergence in distribution
\providecommand{\indep}{\perp \!\!\!\! \perp}              % Independent

% Additional commands
\newcommand\myText[1]{\text{\scriptsize\tabular[t]{@{}c@{}}#1\endtabular}}


% File paths
\newcommand{\figpath}{""}
\newcommand{\tabpath}{""}


\title{Topics in Applied Econometrics}
\subtitle[]{Imperfect compliance}
% Single author
\author[Attila Gyetvai]{Attila Gyetvai}
\institute[]{Duke Economics \\ Summer 2020}
% % Multiple authors
% \author[Gyetvai, Zhu]{Attila Gyetvai\inst{1} \and Maria Zhu\inst{1}}
% \institute[Duke] % 
% {
%  \inst{1} Department of Economics \\ Duke University \and
% }
\date{}


\begin{document}

{
\setbeamertemplate{headline}{}            % Empty header
\setbeamertemplate{footline}{}            % Empty footer  
\begin{frame}
  \titlepage
\end{frame}
}
\addtocounter{framenumber}{-1}

% ----------------------------------------------- % 
\begin{frame}[c]{How to tackle an empirical project}
  \begin{enumerate}
    \addtolength{\itemsep}{0.5\baselineskip}
    \item What causal effects are we interested in?
    \item \textcolor{lightgray}{What ideal experiment would capture this effect?}
    \item \textcolor{lightgray}{What is our identification strategy?}
    \item \textcolor{lightgray}{What is our mode of statistical inference?}
  \end{enumerate}
\end{frame}

\begin{frame}{Treatment status vs.\ assignment}
  \begin{itemize}
    \item Not everyone is treated who is supposed to be
    \item We should separate treatment status \(D\) from assignment to treatment \(Z\)
    \item \(Z_i = 1\) if \(i\) is assigned to be treated \(\big/\) 0 if \(i\) is not assigned to be treated
    \item Note: we haven't yet used the language ``control groups''
    \begin{itemize}
      \item Assignment to control can also be imperfect, just like assignment to treatment
      \item Now we can formalize control units: \(Z_i=0\)
      \item We are typically interested in the difference between treated and untreated units, \\ not treatment and control
      \item If the takeup of the treatment is perfect, all control units are untreated
    \end{itemize}
    \item However, treatment takeup is seldom perfect
  \end{itemize}
\end{frame}

\begin{frame}{Treatment status vs.\ assignment (cont'd)}
  \begin{itemize}
    \item We can think of treatment takeup analogously to potential outcomes: \\
    would the unit be treated if it is assigned to treatment?
    \item \(D_{1i}\): treatment status of \(i\) when assigned to treatment
    \item \(D_{0i}\): treatment status of \(i\) when assigned to control
    \item Observed treatment status:
    \begin{align*}
      D_i &= \begin{cases}
        D_{1i} & \text{ if } Z_i = 1 \\
        D_{0i} & \text{ if } Z_i = 0
      \end{cases}
      \end{align*}
  \end{itemize}
\end{frame}

\begin{frame}{Compliance types}
  \begin{itemize}
    \item The two potential treatment statuses and the two assignments create \\ a \(2 \times 2\) matrix of compliance types

    \vspace*{1em}
    \begin{table}
      \centering
      \begin{tabular}{c|c|c|}
        \multicolumn{1}{c}{} & \multicolumn{1}{c}{\(D_{0i}=1\)} & \multicolumn{1}{c}{\(D_{0i}=0\)} \\
        \cline{2-3}
        \(D_{1i}=1\) & always-takers & compliers     \\
        \cline{2-3}
        \(D_{1i}=0\) & defiers       & never-takers  \\
        \cline{2-3}
      \end{tabular}
    \end{table}
    \item In an ideal experiment, we only have compliers
    \item Also, we typically rule out defiers
  \end{itemize}
\end{frame}

\begin{frame}{Observing compliance types}
  \begin{itemize}
    \item Which compliance types can we observe in the data?
    \item \(2 \times 2\) matrix of compliance types by treatment status and assignment to treatment (no defiers)

    \vspace*{1em}
    \begin{table}
      \centering
      \begin{tabular}{c|c|c|}
        \multicolumn{1}{c}{} & \multicolumn{1}{c}{\(Z_{i}=1\)} & \multicolumn{1}{c}{\(Z_{i}=0\)} \\
        \cline{2-3}
        \(D_{i}=1\) & compliers / always-takers & always-takers \\
        \cline{2-3}
        \(D_{i}=0\) & never-takers  & compliers / never-takers  \\
        \cline{2-3}
      \end{tabular}
    \end{table}
  \end{itemize}
\end{frame}

\begin{frame}{Potential outcomes revisited}
  \begin{itemize}
    \item Potential outcomes can also be thought of in this matrix form
    \item \(i\)'s outcome if their assignment is \(Z_i\) and treatment status is \(D_i\)

    \vspace*{1em}
    \begin{table}
      \centering
      \begin{tabular}{c|c|c|}
        \multicolumn{1}{c}{} & \multicolumn{1}{c}{\(Z_{i}=1\)} & \multicolumn{1}{c}{\(Z_{i}=0\)} \\
        \cline{2-3}
        \(D_{i}=1\) & \(Y_{1i} (1)\) & \(Y_{1i} (0)\) \\
        \cline{2-3}
        \(D_{i}=0\) & \(Y_{0i} (1)\) & \(Y_{0i} (0)\)  \\
        \cline{2-3}
      \end{tabular}
    \end{table}

    \item In short: \(Y_{1i} (Z_i)\), \(Y_{0i} (Z_i)\)
  \end{itemize}
\end{frame}

\begin{frame}{What can we learn with imperfect compliance?}
  \begin{itemize}
    \item We still can compare outcomes of treated vs.\ untreated units
    \item \ldots but we can also compare outcomes of units assigned to treatment vs.\ control
    \begin{block}{Intent-to-treat effect (ITT)}
      \(ITT = \E(Y_{i} \,|\, Z_i=1) - \E(Y_i \,|\, Z_i=0)\)
    \end{block}
    \item In some settings, ITT is more important than ATE
    \item \emph{Example:} mosquito nets \citep*[inspired by][QJE]{Cohen2010}
    \begin{itemize}
      \item Imagine that we equip some Kenyan households with mosquito nets to fend off bugs spreading malaria
      \item Even if some households reject the donation, they benefit from an overall decrease of malaria cases in the area
    \end{itemize}
  \end{itemize}
\end{frame}

\begin{frame}{Treatment effect among compliers}
  \begin{itemize}
    \item The assumptions behind ATE are sometimes unlikely to hold
    \item However, under a set of milder assumptions we can identify the effect \\ among compliers, called LATE
    \begin{block}{Local average treatment effect (LATE)}
      \begin{equation*}
        LATE = \E(Y_{1i} - Y_{0i} \,|\, D_{1i}=1, D_{0i}=0)
      \end{equation*}
    \end{block}
    \begin{itemize}
      \item ``Local:'' in the neighborhood of switching from untreated to treated, for those who comply
    \end{itemize}
    \item However, we must be careful about LATE assumptions
  \end{itemize}
\end{frame}

\begin{frame}{LATE assumptions}
  \begin{enumerate}
    \item Independence: \((Y_{1i}, Y_{0i}, D_{1i}, D_{0i}) \indep Z_i\)
    \begin{itemize}
      \item Assignment to treatment is independent of potential treatment status and potential outcomes
    \end{itemize}
    \item Exclusion: \(Y_{1i} (1) = Y_{1i} (0)\) and \(Y_{0i} (1) = Y_{0i} (0)\)
    \begin{itemize}
      \item Outcomes are only affected by treatment status, not assignment
    \end{itemize}
    \item Monotonicity: \(D_{1i} \geq D_{0i}\)
    \begin{itemize}
      \item If \(i\) is treated when not assigned, they must not be untreated when assigned
      \item In other words: no defiers
    \end{itemize}
  \end{enumerate}
  \begin{itemize}
    \item It is possible to further relax these assumptions 
  \end{itemize}
\end{frame}


\appendix
\begin{frame}[plain,c]
  \centerline{\Large{{\usebeamercolor[fg]{title} Additional Slides}}}
\end{frame}
\addtocounter{framenumber}{-1}
\begin{frame}[allowframebreaks]{References}
  \bibliographystyle{mychicago}
  \linespread{1}
  \begin{footnotesize}
  \bibliography{../refs}
  \end{footnotesize}
\end{frame}


\end{document}
 