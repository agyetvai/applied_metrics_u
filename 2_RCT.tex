\documentclass[aspectratio=169,compress,handout,t,xcolor=table]{beamer}

%! Mandatory packages
\usepackage[utf8]{inputenc}                                %! Character encoding
\usepackage[T1]{fontenc}                                   %! Font encoding
\usepackage[english]{babel}                                %! Language setting
\usepackage{graphicx}

%%%% Frame design
\usetheme{Boadilla}                                        % Design base
\usecolortheme{default}                                    % Color structure base

% Colors
\definecolor{MyBackground}{HTML}{FFFAEE}                   % Cream color with dark text works well for dyslexic readers: FFFAEE
\definecolor{MyBackground}{HTML}{FFFFFF}                   % White for online slides
\setbeamercolor{background canvas}{bg=MyBackground}
\definecolor{MyStructure}{HTML}{005471}
\setbeamercolor{title}{fg=MyStructure}
\setbeamercolor{frametitle}{fg=MyStructure}
\setbeamercolor{structure}{fg=MyStructure}

% Fonts
\usepackage{tgheros}                                       % TG Heros font package
\beamertemplatenavigationsymbolsempty                      % Disable navigation bar

% Enumerate / itemize design
\setbeamertemplate{enumerate items}[circle]
\setbeamertemplate{itemize items}[circle]
\setbeamertemplate{itemize subitem}[triangle]
\setbeamertemplate{section in toc}[sections numbered]
\usepackage{xpatch}                                        % Global itemize/enumerate list separation
\xpatchcmd{\itemize}
  {\def\makelabel}
  {\ifnum\@itemdepth=1\relax
     \setlength\itemsep{2ex}% separation for first level
   \else
     \ifnum\@itemdepth=2\relax
       \setlength\itemsep{1ex}% separation for second level
     \else
       \ifnum\@itemdepth=3\relax
         \setlength\itemsep{0.5ex}% separation for third level
   \fi\fi\fi\def\makelabel
  }
{}
{}


% Footer
\setbeamertemplate{footline}
{
 \hbox{%
  \begin{beamercolorbox}[wd=.2\paperwidth,ht=2.6ex,dp=1ex,center]{section in foot}%
    \usebeamerfont{section in foot}\insertshortauthor
  \end{beamercolorbox}%
  \begin{beamercolorbox}[wd=.75\paperwidth,ht=2.6ex,dp=1ex,center]{section in foot}%
    \usebeamerfont{section in foot}\insertsubtitle
  \end{beamercolorbox}%
  \begin{beamercolorbox}[wd=.05\paperwidth,ht=2.6ex,dp=1ex,center]{subsection in foot}%
    \usebeamerfont{section in foot} \insertframenumber/\inserttotalframenumber
  \end{beamercolorbox}}%

  \vskip0pt%
}
\setbeamertemplate{frametitle}[default][left]              % Push frame title to left
\setbeamertemplate{navigation symbols}{}                   % Remove navigation symbols

\renewcommand\textbullet{\ensuremath{\bullet}}             % Opress textbullet font warning
\usepackage{appendixnumberbeamer}                          % Restart slide numbering for backup slides (\appendix)

% Roadmap
\AtBeginSection[]
{
	\ifnum \value{framenumber}>0
   \begin{frame}[plain]
       \frametitle{Plan for today}
       \tableofcontents[currentsection]
   \end{frame}
   \addtocounter{framenumber}{-1}
  \else
   \fi
}

% Stable preamble with path
% Text packages
\usepackage{enumerate}                                     % Custom enumeration lists
\usepackage{footmisc}                                      % Stable footnotes in headings
\usepackage{natbib}                                        % Bibliography
\usepackage{hyperref}                                      % Hyperreferences
    \hypersetup{
      % colorlinks,
      % citecolor=black,
      % filecolor=black,
      % linkcolor=black,
      urlcolor=MyStructure
    }
\usepackage{multicol}                                      % Multiple columns in text
\usepackage[nointegrals]{wasysym}                          % WASY2 symbols for contradiction \lightning
\usepackage{algorithm}                                     % Pseudocode packages
\usepackage{algpseudocode}
\usepackage{setspace}                                      % Custom line spacing

% Figure and table packages
\usepackage{float}                                         % Figure floats
\usepackage[position=top]{subfig}                          % Subfigures
\usepackage{multirow}                                      % Merged rows in tables
\usepackage{dcolumn}                                       % Custom table delimiters
    \newcolumntype{d}[1]{D{.}{.}{#1}}
    \newcolumntype{m}[1]{D{.}{\$}{#1}}
    \newcommand\hd[1]{\multicolumn{1}{c}{#1}}
\usepackage[labelfont=bf, font=normalsize]{caption}        % Bold captions
    \captionsetup{format=hang}
\usepackage{array}                                         % Reveal table by columns
\usepackage{animate}                                       % Animated figures, GIF-like appearance
\usepackage{media9}                                        % For \mediabutton

% Graphics packages
\usepackage{tikz}                                          % TikZ drawings
\usepackage{pgfplots}                                      % PGFPlots plots
    \pgfplotsset{                                          % PGFPlots options
        cartesian/.style={                                 % Declaring Cartesian plane style
            % Axis alignment
            axis x line=middle,
            axis y line=middle,
            % Axis labels alignment
            every axis x label/.style={at={(current axis.right of origin)},anchor=north west},
            every axis y label/.style={at={(current axis.above origin)},anchor=north east},
            % Legend alignment
            every axis legend/.append style={legend pos=outer north east}
        },
        % Version declaration for compatibility
        compat=1.11
    }
\usetikzlibrary{calc}
    % Node setup for extensive form games
    \tikzset{
        % Two node styles for game trees: solid and hollow
        solid node/.style={circle,draw,inner sep=1.5,fill=black},
        hollow node/.style={circle,draw,inner sep=1.5},
        % Solution nodes
        solidsol node/.style={blue,circle,draw,inner sep=1.5,fill=blue},
        hollowsol node/.style={blue,circle,draw,inner sep=1.5}
    }

% Math packages
\usepackage{amsmath}                                       % AMS math package
\usepackage{amssymb}                                       % Math symbols
\usepackage{amsthm}                                        % Custom theorem environments
\usepackage{units}                                         % Numerical fractions
\usepackage{centernot}                                     % Logical negation in the middle of characters; e.g. not iff
\usepackage{dsfont}                                        % For indicator function: \(\mathds{1}\)

\renewcommand{\qedsymbol}{$\blacksquare$}                  % QED

% Mathematical operators
\DeclareMathOperator{\E}{\mathbb{E}}                       % Expected value
\DeclareMathOperator{\Emax}{\E\!\max}                      % Emax
\DeclareMathOperator{\var}{var}                            % Variance
\DeclareMathOperator{\cov}{cov}                            % Covariance
\DeclareMathOperator{\corr}{corr}                          % Correlation
\DeclareMathOperator{\avar}{avar}                          % Asymptotic variance
\DeclareMathOperator*{\plim}{plim}                         % Probability limit
\DeclareMathOperator{\lag}{lag}                            % Lag operator
\DeclareMathOperator{\rank}{rank}                          % Rank
\DeclareMathOperator{\tr}{tr}                              % Trace
\DeclareMathOperator{\diag}{diag}                          % Diagonal
\DeclareMathOperator{\I}{I}                                % Identity matrix

\providecommand{\abs}[1]{\left\lvert#1\right\rvert}        % Absolute value
\providecommand{\norm}[1]{\left\lVert#1\right\rVert}       % Norm
\providecommand{\ip}[1]{\left\langle#1\right\rangle}       % Inner product
\providecommand{\csp}{\overline{\mathrm{sp}}}              % Closed span
\providecommand{\pto}{\overset{p}{\to}}                    % Convergence in probability
\providecommand{\dto}{\overset{d}{\to}}                    % Convergence in distribution
\providecommand{\indep}{\perp \!\!\!\! \perp}              % Independent

% Additional commands
\newcommand\myText[1]{\text{\scriptsize\tabular[t]{@{}c@{}}#1\endtabular}}


% File paths
\newcommand{\figpath}{""}
\newcommand{\tabpath}{""}


\title{Topics in Applied Econometrics}
\subtitle[]{Randomized experiments}
% Single author
\author[Attila Gyetvai]{Attila Gyetvai}
\institute[]{Duke Economics \\ Summer 2020}
% % Multiple authors
% \author[Gyetvai, Zhu]{Attila Gyetvai\inst{1} \and Maria Zhu\inst{1}}
% \institute[Duke] % 
% {
%  \inst{1} Department of Economics \\ Duke University \and
% }
\date{}


\begin{document}

{
\setbeamertemplate{headline}{}            % Empty header
\setbeamertemplate{footline}{}            % Empty footer  
\begin{frame}
  \titlepage
\end{frame}
}
\addtocounter{framenumber}{-1}

% ----------------------------------------------- % 
\begin{frame}[c]{How to tackle an empirical project}
  \begin{enumerate}
    \addtolength{\itemsep}{0.5\baselineskip}
    \item \textcolor{lightgray}{What causal effects are we interested in?}
    \item What ideal experiment would capture this effect?
    \item \textcolor{lightgray}{What is our identification strategy?}
    \item \textcolor{lightgray}{What is our mode of statistical inference?}
  \end{enumerate}
\end{frame}

\begin{frame}{Consequences of randomization}
  \begin{itemize}
    \item Imagine treatment and control groups with no compliance issues 
    \item Units are \emph{truly} randomly assigned to treatment
    \item What happens to the expected outcomes?
    \begin{align*}
      \E(Y_{1i} \,|\, D_i=1) = \E(Y_{1i} \,|\, D_i=0)
    \end{align*}
    \item But the potential outcome under treatment for the treated is not counterfactual!
    \item Therefore the ATE is simply
    \begin{align*}
      ATE = \E(Y_i \,|\, D_i=1) - \E(Y_i \,|\, D_i=0)
    \end{align*}
    \item Both terms are measurable, thus the ATE is identified
  \end{itemize}
\end{frame}

\begin{frame}{Randomized controlled trials (RCTs)}
  \begin{itemize}
    \item Experimental designs to test the effect of a well-defined treatment
    \item Originated in medicine
    \item \emph{Idea:} ``the only thing different across similar units in treatment/control is chance''
    \item Under perfect randomization, potential outcomes are independent of treatment status
    \begin{align*}
      (Y_{1i}, Y_{0i} \indep D_i = Z_i)
    \end{align*}
  \end{itemize}
\end{frame}

\begin{frame}{RCTs with imperfect compliance}
  \begin{itemize}
    \item Randomizing assignment is relatively simple
    \item However, ensuring random treatment status is much harder
    \item \emph{Example:} Dallas Buyers Club (\href{https://youtu.be/ety5kOjlXoU}{\color{blue}\underline{trailer}})
    \begin{itemize}
      \item HIV/AIDS treatment drug is under FDA testing
      \item The central question: is the drug effective?
      \item Patients are randomly given the drug and placebo
      \item However, compliance is less than perfect\ldots
    \end{itemize}
  \end{itemize}
\end{frame}

\begin{frame}{RCTs with imperfect compliance (cont'd)}
  \begin{itemize}
    \item Under perfect randomization, potential outcomes are independent of treatment assignment
    \begin{align*}
      (Y_{1i}, Y_{0i} \indep Z_i)
    \end{align*}
    \item We can identify ITT
    \begin{align*}
      ITT = \E(Y_{i} \,|\, Z_i=1) - \E(Y_{i} \,|\, Z_i=0)
    \end{align*}
    \item But is ITT enough? Not in the case of HIV/AIDS treatment
  \end{itemize}
\end{frame}

\begin{frame}{RCTs in social sciences}
  \begin{itemize}
    \item ``Gold standard'' of impact assessment
    \item However, rare in practice
    \begin{itemize}
      \item Possibility of implementation
      \item Moral issues
      \item Social cost
      \item Financial cost
      \item Randomized design from day 0
    \end{itemize}
    \item Organizations driving the experimental movement
    \begin{itemize}
      \item J-PAL, Gates Foundation, IMF, World Bank
    \end{itemize}
  \end{itemize}
\end{frame}

\begin{frame}{In the absence RCTs}
  \begin{itemize}
    \item More often than not, you will not be able to run your own experiment
    \item Designs providing ``as-good-as-random'' assignment
    \begin{itemize}
      \item Natural experiments
      \item Quasi-experiments
    \end{itemize}
    \item However, we must be careful with the underlying assumptions
    \begin{itemize}
      \item \emph{Example:} earthquakes are random, effects may not be
    \end{itemize}
    \item We can learn from nonexperimental settings too
    \item But: always think of the ideal experiment
  \end{itemize}
\end{frame}

\end{document}
 