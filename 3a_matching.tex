\documentclass[aspectratio=169,compress,handout,t,xcolor=table]{beamer}

%! Mandatory packages
\usepackage[utf8]{inputenc}                                %! Character encoding
\usepackage[T1]{fontenc}                                   %! Font encoding
\usepackage[english]{babel}                                %! Language setting
\usepackage{graphicx}

%%%% Frame design
\usetheme{Boadilla}                                        % Design base
\usecolortheme{default}                                    % Color structure base

% Colors
\definecolor{MyBackground}{HTML}{FFFAEE}                   % Cream color with dark text works well for dyslexic readers: FFFAEE
\definecolor{MyBackground}{HTML}{FFFFFF}                   % White for online slides
\setbeamercolor{background canvas}{bg=MyBackground}
\definecolor{MyStructure}{HTML}{005471}
\setbeamercolor{title}{fg=MyStructure}
\setbeamercolor{frametitle}{fg=MyStructure}
\setbeamercolor{structure}{fg=MyStructure}

% Fonts
\usepackage{tgheros}                                       % TG Heros font package
\beamertemplatenavigationsymbolsempty                      % Disable navigation bar

% Enumerate / itemize design
\setbeamertemplate{enumerate items}[circle]
\setbeamertemplate{itemize items}[circle]
\setbeamertemplate{itemize subitem}[triangle]
\setbeamertemplate{section in toc}[sections numbered]
\usepackage{xpatch}                                        % Global itemize/enumerate list separation
\xpatchcmd{\itemize}
  {\def\makelabel}
  {\ifnum\@itemdepth=1\relax
     \setlength\itemsep{2ex}% separation for first level
   \else
     \ifnum\@itemdepth=2\relax
       \setlength\itemsep{1ex}% separation for second level
     \else
       \ifnum\@itemdepth=3\relax
         \setlength\itemsep{0.5ex}% separation for third level
   \fi\fi\fi\def\makelabel
  }
{}
{}


% Footer
\setbeamertemplate{footline}
{
 \hbox{%
  \begin{beamercolorbox}[wd=.2\paperwidth,ht=2.6ex,dp=1ex,center]{section in foot}%
    \usebeamerfont{section in foot}\insertshortauthor
  \end{beamercolorbox}%
  \begin{beamercolorbox}[wd=.75\paperwidth,ht=2.6ex,dp=1ex,center]{section in foot}%
    \usebeamerfont{section in foot}\insertsubtitle
  \end{beamercolorbox}%
  \begin{beamercolorbox}[wd=.05\paperwidth,ht=2.6ex,dp=1ex,center]{subsection in foot}%
    \usebeamerfont{section in foot} \insertframenumber/\inserttotalframenumber
  \end{beamercolorbox}}%

  \vskip0pt%
}
\setbeamertemplate{frametitle}[default][left]              % Push frame title to left
\setbeamertemplate{navigation symbols}{}                   % Remove navigation symbols

\renewcommand\textbullet{\ensuremath{\bullet}}             % Opress textbullet font warning
\usepackage{appendixnumberbeamer}                          % Restart slide numbering for backup slides (\appendix)

% Roadmap
\AtBeginSection[]
{
	\ifnum \value{framenumber}>0
   \begin{frame}[plain]
       \frametitle{Plan for today}
       \tableofcontents[currentsection]
   \end{frame}
   \addtocounter{framenumber}{-1}
  \else
   \fi
}

% Stable preamble with path
% Text packages
\usepackage{enumerate}                                     % Custom enumeration lists
\usepackage{footmisc}                                      % Stable footnotes in headings
\usepackage{natbib}                                        % Bibliography
\usepackage{hyperref}                                      % Hyperreferences
    \hypersetup{
      % colorlinks,
      % citecolor=black,
      % filecolor=black,
      % linkcolor=black,
      urlcolor=MyStructure
    }
\usepackage{multicol}                                      % Multiple columns in text
\usepackage[nointegrals]{wasysym}                          % WASY2 symbols for contradiction \lightning
\usepackage{algorithm}                                     % Pseudocode packages
\usepackage{algpseudocode}
\usepackage{setspace}                                      % Custom line spacing

% Figure and table packages
\usepackage{float}                                         % Figure floats
\usepackage[position=top]{subfig}                          % Subfigures
\usepackage{multirow}                                      % Merged rows in tables
\usepackage{dcolumn}                                       % Custom table delimiters
    \newcolumntype{d}[1]{D{.}{.}{#1}}
    \newcolumntype{m}[1]{D{.}{\$}{#1}}
    \newcommand\hd[1]{\multicolumn{1}{c}{#1}}
\usepackage[labelfont=bf, font=normalsize]{caption}        % Bold captions
    \captionsetup{format=hang}
\usepackage{array}                                         % Reveal table by columns
\usepackage{animate}                                       % Animated figures, GIF-like appearance
\usepackage{media9}                                        % For \mediabutton

% Graphics packages
\usepackage{tikz}                                          % TikZ drawings
\usepackage{pgfplots}                                      % PGFPlots plots
    \pgfplotsset{                                          % PGFPlots options
        cartesian/.style={                                 % Declaring Cartesian plane style
            % Axis alignment
            axis x line=middle,
            axis y line=middle,
            % Axis labels alignment
            every axis x label/.style={at={(current axis.right of origin)},anchor=north west},
            every axis y label/.style={at={(current axis.above origin)},anchor=north east},
            % Legend alignment
            every axis legend/.append style={legend pos=outer north east}
        },
        % Version declaration for compatibility
        compat=1.11
    }
\usetikzlibrary{calc}
    % Node setup for extensive form games
    \tikzset{
        % Two node styles for game trees: solid and hollow
        solid node/.style={circle,draw,inner sep=1.5,fill=black},
        hollow node/.style={circle,draw,inner sep=1.5},
        % Solution nodes
        solidsol node/.style={blue,circle,draw,inner sep=1.5,fill=blue},
        hollowsol node/.style={blue,circle,draw,inner sep=1.5}
    }

% Math packages
\usepackage{amsmath}                                       % AMS math package
\usepackage{amssymb}                                       % Math symbols
\usepackage{amsthm}                                        % Custom theorem environments
\usepackage{units}                                         % Numerical fractions
\usepackage{centernot}                                     % Logical negation in the middle of characters; e.g. not iff
\usepackage{dsfont}                                        % For indicator function: \(\mathds{1}\)

\renewcommand{\qedsymbol}{$\blacksquare$}                  % QED

% Mathematical operators
\DeclareMathOperator{\E}{\mathbb{E}}                       % Expected value
\DeclareMathOperator{\Emax}{\E\!\max}                      % Emax
\DeclareMathOperator{\var}{var}                            % Variance
\DeclareMathOperator{\cov}{cov}                            % Covariance
\DeclareMathOperator{\corr}{corr}                          % Correlation
\DeclareMathOperator{\avar}{avar}                          % Asymptotic variance
\DeclareMathOperator*{\plim}{plim}                         % Probability limit
\DeclareMathOperator{\lag}{lag}                            % Lag operator
\DeclareMathOperator{\rank}{rank}                          % Rank
\DeclareMathOperator{\tr}{tr}                              % Trace
\DeclareMathOperator{\diag}{diag}                          % Diagonal
\DeclareMathOperator{\I}{I}                                % Identity matrix

\providecommand{\abs}[1]{\left\lvert#1\right\rvert}        % Absolute value
\providecommand{\norm}[1]{\left\lVert#1\right\rVert}       % Norm
\providecommand{\ip}[1]{\left\langle#1\right\rangle}       % Inner product
\providecommand{\csp}{\overline{\mathrm{sp}}}              % Closed span
\providecommand{\pto}{\overset{p}{\to}}                    % Convergence in probability
\providecommand{\dto}{\overset{d}{\to}}                    % Convergence in distribution
\providecommand{\indep}{\perp \!\!\!\! \perp}              % Independent

% Additional commands
\newcommand\myText[1]{\text{\scriptsize\tabular[t]{@{}c@{}}#1\endtabular}}


% File paths
\newcommand{\figpath}{""}
\newcommand{\tabpath}{""}


\title{Topics in Applied Econometrics}
\subtitle[]{Matching}
% Single author
\author[Attila Gyetvai]{Attila Gyetvai}
\institute[]{Duke Economics \\ Summer 2020}
% % Multiple authors
% \author[Gyetvai, Zhu]{Attila Gyetvai\inst{1} \and Maria Zhu\inst{1}}
% \institute[Duke] % 
% {
%  \inst{1} Department of Economics \\ Duke University \and
% }
\date{}


\begin{document}

{
\setbeamertemplate{headline}{}            % Empty header
\setbeamertemplate{footline}{}            % Empty footer  
\begin{frame}
  \titlepage
\end{frame}
}
\addtocounter{framenumber}{-1}

% ----------------------------------------------- % 
\begin{frame}[c]{How to tackle an empirical project}
  \begin{enumerate}
    \addtolength{\itemsep}{0.5\baselineskip}
    \item \textcolor{lightgray}{What causal effects are we interested in?}
    \item \textcolor{lightgray}{What ideal experiment would capture this effect?}
    \item What is our identification strategy?
    \item \textcolor{lightgray}{What is our mode of statistical inference?}
  \end{enumerate}
\end{frame}

\begin{frame}{Applying the framework to observational data}
  \begin{itemize}
    \item Imagine we have data on outcomes of two groups
    \item We did not ensure randomization
    \item Biggest fear: selection into treatment
    \item What if units select into treatment based on observable characteristics?
    \item \emph{Example:} college wage premium
    \begin{itemize}
      \item Treatment: getting a college degree
      \item Control: no degree
      \item Who chooses to go for a degree is not random
      \item But imagine two identical persons (gender, age, ability, parents etc.), with / without a degree
    \end{itemize}
  \end{itemize}
\end{frame}

\begin{frame}{Unconfoundedness}
  \begin{itemize}
    \item Assignment to treatment is not random
    \item \emph{But:} assignment to treatment is as good as random, for comparable units
    \item In other words, no observable characteristics confound the effect of the treatment
    \begin{align*}
      (Y_{1i}, Y_{0i}) \indep D_i \,|\, X_i
    \end{align*}
    \item AKA selection on observables
    \item \emph{Problem:} the unconfoundedness assumption cannot be tested
    \begin{itemize}
      \item Why? Because of the fundamental problem of causal inference
    \end{itemize}
  \end{itemize}
\end{frame}

\begin{frame}{Core idea behind matching}
  \begin{itemize}
    \item Imagine we observe all possible confounding factors (\(X\))
    \item Then, we can compare outcomes for people with the exact same characteristics
    \begin{equation*}
      ATE(x) = \E(Y_i \,|\, D_i=1, X_i=x) - \E(Y_i \,|\, D_i=0, X_i=x)
    \end{equation*}
    \item Assumptions:
    \begin{enumerate}
      \item Overlap: we need enough units in both groups with the same characteristics
      \item Unconfoundedness: we need to observe all confounding \(X\)'s
    \end{enumerate}
    \item Let's look at some matching procedures
  \end{itemize}
\end{frame}

\begin{frame}{Exact matching}
  \begin{itemize}
    \item Match treatment units to control units by perfectly aligning \(X\)'s
    \item Recall the college wage premium example
    \item But: are these matches really exact?
    \item College wage premium
    \begin{enumerate}
      \item Overlap: we likely do not have enough identical grads/non-grads
      \item Unconfoundedness: we likely do not observe every single confounder
    \end{enumerate}
    \item Another example: twin studies
    \begin{enumerate}
      \item Overlap: no problem
      \item Unconfoundedness: depends
    \end{enumerate}
  \end{itemize}
\end{frame}

\begin{frame}{Propensity score matching}
  \begin{itemize}
    \item The overlap requirement is hard to fulfill in practice
    \item \emph{Idea:} match units based on an overall score
    \item Propensity score: probability of being treated, given \(X\)
    \begin{equation*}
      p(X_i) = \Pr (D_i=1 \,|\, X_i)
    \end{equation*}
    \item Reduces the dimension of \(X\) to one \(\implies\) overlap is better
    \item Unconfoundedness assumption changes
    \begin{equation*}
      (Y_{1i}, Y_{0i}) \indep D_i \,|\, X_i \quad \implies \quad (Y_{1i}, Y_{0i}) \indep D_i \,|\, p(X_i)
    \end{equation*}
    \begin{itemize}
      \item Intuitively: \(p(X)\) fully captures the relationship between \(X\) and \(D\)
      \item Technically: \citet*[][Biometrika]{Rosenbaum1983}
    \end{itemize}
  \end{itemize}
\end{frame}

\begin{frame}{Propensity score matching in practice}
  \begin{enumerate}
    \item Estimate the propensity score model
    \begin{itemize}
      \item Linear probability model: \(D_i = X_i \beta + u_i\)
      \item Logit: \(\Pr(D_i=1) = \Lambda(X_i \beta)\)
      \item Probit: \(\Pr(D_i=1) = \Phi(X_i \beta)\)
    \end{itemize}
    \item Forecast \(\hat{p}(X_i)\)
    \item Check overlap of propensity scores
    \item Form matches
    \begin{itemize}
      \item Nearest neighbor: match closest units, with/without replacement
      \item Block: match all units within a distance
      \item Many fancy techniques; report various ones
    \end{itemize}
    \item Check overlap of confounders!
    \begin{itemize}
      \item Similar propensity score does not automatically mean similar \(X\)'s
    \end{itemize}
    \item Calculate treatment effects
  \end{enumerate}
\end{frame}

\begin{frame}{What can we learn from matching? }
  The UK New Deal for Lone Parents \citep*[][IZA DP]{Dolton2011}
  \begin{itemize}
    \item Lone parents receive a personal advisor to find jobs or work longer
    \item Participation is voluntary \(\implies\) selection into treatment
    \item \emph{Idea:} match participants to nonparticipants on benefit history
    \item Benefit takeup in six consequent periods generates ``strings'' \\ (101101, 000110 \ldots)
    \item Match on \(2^6\) strings and other characteristics
  \end{itemize}
  \begin{quote}
    \footnotesize ``[L]agged outcomes correlate strongly both with other observed determinants of participation and outcomes and with otherwise unobserved determinants such as tastes for leisure, particular family obligations such as seriously ill or disabled parents or children and so on. Thus, in our view, conditioning on these histories goes a long way toward solving the selection problem.''
  \end{quote}
\end{frame}

\appendix
\begin{frame}[plain,c]
  \centerline{\Large{{\usebeamercolor[fg]{title} Additional Slides}}}
\end{frame}
\addtocounter{framenumber}{-1}
\begin{frame}[allowframebreaks]{References}
  \bibliographystyle{mychicago}
  \linespread{1}
  \begin{footnotesize}
  \bibliography{../refs}
  \end{footnotesize}
\end{frame}


\end{document}
 